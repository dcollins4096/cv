

\noindent{\large\bf Current Position and  Address:}
 Full Professor                                                                \\
 Department of Physics, Florida State University, Tallahassee,FL 32312         \\
 E-mail: pah@astro.physics.fsu.edu , Tel.: 850-644-5567                       \\
 Citizenship: United States                                                   \\



\noindent {\large\bf Expertise:}                                                              \\  
 radiation and radiation hydrodynamics,  non-LTE problems, transport problems,                    \\  
 nuclear processes,  molecule formation, supernovae, distance determinations, cosmology                \\  

                                                                                                              
\noindent {\large\bf Education:}        \\  
 Master in Physics (February 1984), University of Heidelberg, FRG     \\  
 PhD in Astronomy (May 1986). University of Heidelberg, FRG                \\  
 Dr. rer. nat habil./Venia legendi (July 1992). University of Munich, FRG       \\  


\noindent {\large\bf Professional Membership:}    \\
 American Astronomical Society                      \\
 International Astronomical Union                     \\
 American Physical Society/Fellow of the APS           \\

\noindent {\large\bf Professional Experience:}        \\
\noindent
 Full Professor  (August 2012 till now),                
 Dept. of Physics, Florida State University, Tallahassee \\
 Associate Professor  (August 2006 till July 2012),
 Dept. of Physics, FSU   \\
 Senior Research  Scientist (August 2004 till July 2006), Research  Scientist (August 1996 - August 2004),
Dept. of Astronomy, University of Texas at Austin \& McDonald Observatory   \\
 Visiting Scientist (SS 1997 and SS 1998), Dept. for Theoretical Physics, U. Basel, Switzerland  \\
  Heisenberg fellow \& Lecturer (July 1993 - August 1996),
Harvard University \& CfA    \\
 Research Assistent (October 1992 - June 1993), University of Munich, FRG   \\
 Research Assistent (October 1986 - September 1992), MPA, Garching, FRG       \\

\noindent{\large\bf Publications:} Total:336, Refereed: 135,
\noindent{h-index:} 46 (ADS)                                      \\

\noindent {\large\bf Some of the Relevant Publications}      \\
\noindent
- Patat F., Hoeflich P., Baade D.,  Maund J., Wang L. Wheeler J.C. 2012,
{\sl  VLT Spectropolarimetry of the Type Ia SN 2005ke. A step towards understanding subluminous events},
 A\&A, 545,7                                                   \\
\noindent
- Sadler, B., H"oflich P., Baron E., Krisciunas K. Folatelli G., Hamuy M., Khokhlov A., Phillips M., Suntzeff L., Wang L.  2012,
{\sl  Constraining the Properties of SNe~Ia Progenitors}, in: Binary Path to SNeIa, eds. DiStephano \& Orio, CUP \& http://lanl.arxiv.org/abs/1109.3629    //
\noindent
- Hoeflich, P.; Krisciunas, K.; Khokhlov, A. M.; +6 2010, {\sl Secondary Parameters of Type Ia Supernova Light Curves}, ApJ, 710, 444                    \\
\noindent
- Hoeflich P. 2009, {\sl Multi-dimensional Radiation Transport in Rapidly Expanding Envelopes}, in:
Recent Directions in Radiation Hydrodynamics, eds. Hubeney et al., AIP, p. 161                                                \\
\noindent
- Quimby, R., H{oe}flich, P., Wheeler, J.~C.\ 2007.\ SN 2005hj: Evidence for Two Classes of
Normal-Bright SNe Ia and Implications for Cosmology.  ApJ, 666, 1083.                                                           \\
\noindent
- Fesen R., Hoeflich P., Hamilton A., +4. 2007, {\sl Type Ia Supernova: Hubble Space Telescope Images of the SN 1885 Remnant}, ApJ 658, 396   \\
\noindent
- Hoeflich P., 2006,
{\sl  Physics of Thermonuclear Supernovae \& Cosmology},
 ApJ , Nuclear Physics A 777, 579                                                                               \\
\noindent
- Hoeoflich P., Gerardy C., Nomoto K., Motohara K., Fesen R., Maeda K., Ohukubo T., Tominaga N.  2004, 
{\sl Signature of Electron Capture in Iron-rich Ejecta of SN 2003du}, ApJ 617, 1258               \\
\noindent
- Hoeflich P., Gerardy C., Fesen R.,Sakai S. 2002, {\sl Infrared Spectra of the Subluminous Type Ia Supernova SN 1999by}, ApJ 575, 1007     \\
\noindent
- Hoeflich, P.; Khokhlov, A. M.; Wheeler, J. C. 1995, DD-Models for normal and subluminous SNeIa: Absolute brightness, light curves, and molecule formation, ApJ 444, 831 \\
 
\noindent {\large\bf Other Significant Publications}      \\
\noindent
-Hoeflich P., Stein Y. 2002,
{\sl On the Thermonuclear Runaway in SNe~Ia: How to run away?},
{ApJ 568}, 771                                              \\
\noindent 
- Hoeflich P., Wheeler J.C.    Thielemann F.K. 1998,
{\sl Type Ia Supernovae: Influence of the Initial Composition on the 
Nucleosynthesis, Light Curves, Spectra and Consequences for 
of $q_o$, $\Omega _o$ \& $\Lambda$}, { ApJ}, { 495}, 617      \\                                         
\noindent 
- Hoeflich P., Khokhlov A. 1996, {Explosion Models of SNeIa\& $H_o$, 1996 { ApJ}, 457, 500  \\
\noindent 
- Hoeflich P., Khokhlov A., Wheeler J., Phillips M., Suntzeff N. Hamuy M.  1996, {The M(dm15) relation: Theory and Observations,  {ApJ}, 472, 81   \\

\noindent
{\large\bf Synergetic Activities and Student Research during the last 2 Years}   \\
- Hoeflich is director of the FSU planetarium which is dedicated to outreach
 (http://www.physics.fsu.edu/outreach/Planetarium/default.htm).                    \\
- Hoeflich has guided yearlong research projects for undergraduate, graduate and summer students
 and, currently, supervises three undergraduate and three graduate students.          \\
- on a regular basis, Hoeflich gives lectures at a number of schools to promote Astronomy on an
international level.                                                                    \\

\noindent
{\large\bf Main Collaborators during the last four years:}   \\
D. Baade (ESO/Garching),
E. Baron (U. Oklahoma);
B. Berg  (FSU, professor);
E. Bravo(U. Barcellona);
A. Chieffi (Rome/Italy),
A. Clocchiatti (U. Catholica, Chile);
  I. Dominguez (Grenada/Spain),
R. Fesen (Dartmouth),C. Gerardy (FSU), 
M. Hamuy (CTIO),
J. Isern (U. Barcellona);
A. Khokhlov (U.Chicago), 
R. Kotak (U of Belfast/IR);
K. Krisciunas (Texas A\& M),
J. Maund (U. of Belfast/Irland);
P. Meikle (Imperial College/London),
F. Patat (ESO/Garching);
M. Phillips (Carnegie);
 E. Livne \& Y. Stein (Jerusalem/Israel), N. Suntzeff (Texas A\&M)
O. Straniero (Teramo/Italy),  F.K. Thielemann (U. Basel/Switzerland),
 L. Wang (Texas A\&M),
C. Wheeler (U Texas)                                           \\
\noindent
 {\large\bf Thesis Advisor:} Prof. B. Baschek, ITA, U. of Heidelberg, Germany  \\
 {\large\bf Postdoctorial advisors:}R. Kippenhahn \& W. Hillebrandt,
   MPI/Garching, 1986-1992}                                             \\
 {\large\bf Thesis advisor or postdoctorial advisor of} A. Howell(Co), H. Marion (Co), R. Quimby(Co), A. Zighlo,
  P. Dragolin, R. Dungan, B. Penny, J. Mitchell, B. Sadler, S. Yuan}      \\




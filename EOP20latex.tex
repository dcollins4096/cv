\documentstyle[12pt]{article}
\addtolength{\topmargin}{-.5in}
\addtolength{\textheight}{2.5cm}
\addtolength{\oddsidemargin}{-.75in}
\addtolength{\evensidemargin}{-.75in}
\addtolength{\textwidth}{.5in}
\begin{document}

%\vskip 1 in
\centerline{EVIDENCE OF PERFORMANCE}
%\centerline{(compressed format)}
\bigskip
\centerline{YEAR: \underbar{2020}}
\bigskip
\noindent Name: \underbar {David Collins }
\hbox{\ \ \ \ \ \ \ \
\ \ \ \ \ \ \ \ \ \ \ \ \ \ \ \ \ \ \ \ \ \ \ \ \ }
Rank: \underbar
{Associate Prof. }

%\bigskip
\vskip 20pt

\noindent A. YOUR MOST SIGNIFICANT CONTRIBUTIONS FOR THIS YEAR.
Please select from this document what you consider were your most
significant contributions in each of the areas of teaching, research
and service. PLEASE LIST THEM IN BULLET FORM AND ICLUDE A BRIEF 
DESCRIPTION AND/OR SUPPORTIVE INFORMATION AS SUB-BULLETS. 
If there is something important that you would like to mention which
lies outside these categories, please
place it in a category named ``Other'' at the bottom of this page. \\ \\

{\bf Please include in each section below your efforts and activities related to teaching, research, and service in order to meet the demands during the COVID-19 health emergency. Additional detail and guidance is provided in the College Memorandum on the Guidelines for 2020 Faculty Evaluations.}


\vskip 10pt
\noindent
Teaching:
\begin{itemize}
    \item AST-4419/AST 5418, Extragalactic Astronomy (Spring)
    \item PHY2053C, Recitation instruction for the large class.
\end{itemize}
\vskip 30pt

\noindent
Research:
\begin{itemize}
    \item \emph{Search Polarized Foregrounds of the Cosmic Microwave
        Background}:  In collaboration with Kevin Huffenberger.  Also with grad
        Kye Stalpes.  In order
        to observe the polarized signature of inflation in the cosmic microwave
        background, we are studying the gas in the local galaxy that
        contaminates our observation of that signal.  This is a suite of simulations
        that spans a range of conditions typical of the interstellar gas that
        makes up the bulk of the ``contamination'' of the CMB signal. Presently
        there is no analytical theory that relates gas properties to their
        signature when observed in the microwave.  These
        simulations will allow us to relate the statistics of the polarized gas
        (the power spectra in $E$ and $B$ modes) to properties of the gas that
        we can infer from theory and simulations (spectra of density, velocity,
        and magnetic field).  
    \item \emph{Magnetic Fields in Type Ia Supernovae} In collaboration with
        Peter Hoeflich.  Also with Boyan Hristov, now at University of Alabama,
        Huntsville.  We are examining the impact of magnetic fields on
        nuclear burning fronts and light curves.  The result is a measurement
        that strongly suggests that magnetic fields are a necessary component of
        modeling Type Ia.  We infer magnetic field strengths of $10^9$G in some
        observed Type Ia.  This could have impact on use of supernovae as
        standard candles.
    \item \emph{Magnetic Fields in Star Forming Clouds}
        With grads Dan Le, Luz Jiminez-Vela, Bra\u no Rabatin.  This is a series of
        projects aimed at studying the formation of stars from the magnetized
        plasma of interstellar space.  We have run some novel pseudo-lagrangian
        simulations of the gas to measure what gas collapses, and how long it
        takes.  There are several analytic theories in the literature, and we
        aim to test the theories against simulations of minimal idealized star
        forming clouds.  Dan Le is examining the evolution of star forming
        clumps.  Luz Jiminez-Vela is studying the statistics of the magnetic
        field in the collapsing gas.  Bra\u no Rabatin has a beautiful analytic
        model of the internal and kinetic energies of supersonic turbulence,
        which characterizes star forming clouds.

    \item \emph{Star Formation at the National Ignition Facility}  
        Collaboration with Lawrence Livermore National Lab, University of
        Z\"urich, and Australian National University. New this year.  We have been awarded two
        shots at the National Ignition Facility at LLNL to study the nature of
        supersonic turbulence in star forming clouds.  The study will measure
        the ratio of variance in density fluctuations to the variance in
        velocity fluctuations in gas that has been shocked.  Theory predicts
        that this ratio should be 1/3, though dependant on the nature of the
        turbulence.  This ratio is an essential ingredient of star formation
        theory.   We're using NIF to get large enough shock energies to mimic
        the Mach 10 shocks in star forming clouds.

\end{itemize}
\vskip 30pt

\noindent
Service:
\begin{itemize}
	\item Qual Committee.  Writing and grading of the test.
	\item Web Committee (Chair) I supervise the public face of the department,
        and ensure that it is useful and inviting for both members of the
        department as well as the visiting public.  During 2020 I finished
        reworking the directory to be easier to edit and maintain and more fault
        tolerant.  Many minor changes over the year.
    \item PAI Committee (Chair)     
    \item Hiring Committee, FRIB Theory position.  
	\item Saturday Morning Physics.  
\end{itemize}	
\vskip 30pt



\noindent B. PAPERS PUBLISHED THIS YEAR. Please list by citation all
papers of which you were an author that appeared during 2020.
(Designate whether refereed, conference proceeding, abstract etc.)
\bigskip

All are refereed publications

\begin{enumerate}
    \item ``The Power Spectra of Polarized, Dusty Filaments'', Huffenberger, K.; Rotti, A., Collins, D. C. ApJ, 2020, 889, 31
   \item ``The Catalogue for Astrophysical Turbulence Simulcations (CATS)'' Burkhart, B.; Appel, S. M.; Bialy, S.; Cho, J.; Christensen, A. J.; Collins, D.; Federrath, C.; Fielding, D. B.; Finkbeiner, D.; Hill, A. S.; Ibáñez-Mejía, J. C.; Krumholz, M. R.; Lazarian, A.; Li, M.; Mocz, P.; Mac Low, M. -M.; Naiman, J.; Portillo, S. K. N.; Shane, B.; Slepian, Z. Yuan, Y.;
\end{enumerate}

\noindent C. PAPERS IN PRESS. Please list by citation all such
papers that were accepted but did not appear during 2020.
\bigskip


\noindent D. PAPERS SUBMITTED. Please list all papers that were
submitted by you but were not yet accepted during 2020.
\bigskip

\begin{enumerate}
\item ``Physics of Thermonuclear Explosions: Magnetic Field Effects on Deflagration Fronts and Observable Consequences'', Hristove, B; Hoeflich, P; Collins, D. C. submitted to ApJ.
\end{enumerate}

\noindent E. INDIVIDUAL TALKS. Please list all conference, symposia,
colloquia, seminar and other individual talks that you gave during
2020.  (Designate invited talks.)
\bigskip
%
Due to covid, I did not give any research related talks in 2020.


\noindent F. GRANT FUNDING. Please indicate your research funding
for 2020. Include any submitted  (even if not funded) or pending
proposals.   {\bf Include the FSU Project Number.}
\bigskip
%
{\bf Current}
\begin{enumerate}
    \item ``Magnetic Fields in the Formation of Molecular Clouds,
Filaments, and Cores''.  NSF AAG. FSU: 037693.   09/01/2016 - 08/31/2021 \$298,492
\item  ``Modeling CMB polarization foregrounds and their isotropy violation''
    NASA ATP.  01/08/2017 - 01/07/2022 FSU: 100269 \$428,043.00
\item ``Signatures of Type Ia Supernovae Explosions and their Cosmological
    Implications'' NSF AAG FSU: 039518 10/01/2017 - 09/31/2021 \$460,498
\item ``CMB Polarization Foreground Effects on B-modes and Lensing'' NSF FSU:
    100269 02/01/21-01/31/24 \$533,715
\end{enumerate}

{\bf Pending}
\begin{enumerate}
    \item ``Mitigating Galactic Foreground Contamination for CMB-S4 Lensing and
        Delensing'' DOE Cosmic Microwave Background S-4 funding opportunity,
        \$683,340
    \item ``Unravelling the Radio Sky'' NASA ADAP \$300,000
    \item ``Collaborative proposal: Galactic and circumgalactic magnetic fields
        in Milky Way-like galaxies'' \$300,000
\end{enumerate}


\noindent G. DEPARTMENTAL COMMITTEE SERVICE. Please list all
departmental committees on which you served during 2020.  Also
please indicate your committee assignment for this (spring)
semester.
\bigskip
%
\begin{enumerate}
    \item Web and Newsletter Committee, Co-Chair.  Presently I oversee the web
        pages, primarily the front page at physics.fsu.edu.  During 2020 I
        finalized the upgrade of the department directory, among numerous
        regular upkeep changes.  This summer I plan to revamp the research page,
        as well as finish getting FSU IDs on the directory page (this took
        longer than I would have liked)
    \item Qual committee
    \item Saturday Morning Physics (did not happen this year)
    \item PAI award committee (Chair) I collect the nominations for the award, and
        decide with the other members of the committee which of our outstanding
        colleagues gets the award.
    \item Hiring Committee, FRIB Theory position.  Evaluated some of the
        proposals as they came in.  
\end{enumerate}

\noindent H. UNIVERSITY AND SUS SERVICE. Please list all University
and SUS committees, task forces, and governing bodies on which you
served during 2020.
\bigskip
%

\noindent I. INTERNATIONAL, NATIONAL, AND REGIONAL  SERVICE. Please
list all international, national, and regional bodies on which you
served, together with any reviewing activities (journals,
proposals), during 2020.
\bigskip

\noindent J. OFFICES HELD. Please list all offices in organizations
related to your activity as a professional faculty member held by
you during 2020.
\bigskip

\noindent K. HONORS RECEIVED. Please list all professional honors
received by you during 2020.
\bigskip

\noindent L. GRADUATE DEGREES AWARDED. Please list all graduate
students who received degrees under your direction during 2020.
\bigskip

\noindent M. GRADUATE STUDENT DIRECTION. Please list all graduate
students whose work you directed during 2020 but who did not receive
degrees. {\bf Include date joined group, prospectus (expected/passed) date, 
and expected graduation date.}

\begin{enumerate}
    \item Le, Dan K. Joined Spring 2015.  Prospectus December 2018.  Expected
        graduation Spring 2022.
    \item Stalpes, Kye  Joined Spring 2016.  Prospectus December 2018. Expected
        graduation Spring 2022.
    \item Jimenez-Vela, Luz.  Joined Spring 2016 as a Master's student, through
        the \emph{NSF Bridge} program.  Joined the PhD program Fall 2018.
        Expected prospectus Fall 2021.  
    \item Rabatin, Branislav (Bra\u no).  Joined Spring 2019.  Expected
        prospectus Fall 2021.
    \item Kanai, Toshiaki (Co-advisor.  Primary Advisor is Wei Guo, in
        Mechanical Engineering).  Joined Fall 2020.  Expected prospectus Fall
        2021.
\end{enumerate}
Graduate supervision during covid was reasonably effective, owing to the fact
that all of our work was already done at supercomputers across the country,
we're used to being online.  Overall progress of everyone involved was much
slower than normal due to the multitude of distractions due to covid.  

\bigskip

\noindent N. GRADUATE STUDENT COMMITTEES. Please list all graduate
students on whose supervisory committees you served but for whom you
were not Major Professor during 2020. {\bf Include name of Major Professor 
and department if outside of Physics. }
\bigskip

\begin{enumerate}
    \item Bloor, Erica (Jeremiah Murphy)
    \item Garcia, Carlos (Stephen McGill)
    \item Lakey, Vincent (Kevin Huffenberger)
    \item Yavuz, Oz (Irinel Chiorescu)
    \item Knorr, Erica (Ken Hanson, Chemistry)
    \item Rosales, Daniel (Kevin Speer, GFDI)
\end{enumerate}

\noindent O. UNDERGRADUATE SUPERVISION. Please list all
undergraduate students whose research activities you directed during
2020.  {\bf Include a research title or brief description. }
\bigskip

\begin{enumerate}
    \item Schoedel, Douglas (Research assistant)
    \item Strack, Jacob (Honors thesis committee)
\end{enumerate}
Undergraduate supervision was identical to graduate supervision.

\noindent P. CLASSES TAUGHT. Please list all classes you taught
during 2020 and what you are teaching this semester.  Include the
number of students in each class.
\bigskip


\begin{itemize}
    \item AST-4419/AST 5418, Extragalactic Astronomy (Spring).  10 students.
    \item PHY2053C, Recitation instruction for the large class. 510 students.
\end{itemize}
Both classes were remote.

AST4419 was delivered by zoom in a standard lecture format.  Lectures were
recorded for students with connectivity problems and put online.  Discussion was
facilitated by regular reading responses and questions submitted by the
students.  Overall the experience was less effective than a normal year due to
the additional complications of zoom, scanning of handwritten work, grading
through the Canvas tools (which limit the amount of feedback I can provide).  

PHY2053C was also delivered by zoom.  This class is quite large.  Recitation
sections were 70 and 100 students.  Small group sessions were enabled with zoom
breakout rooms.  Overall the learning was substantialy reduced and stress levels
were much higher due to the extra technological barriers to every function, and
inability to monitor multiple breakout rooms simultaneously.  Hopefully we will
not have another semester like that.

\noindent Q. DIS SUPERVISION. Please list all DIS (Directed
Individual Study) students you directed during 2020.
\bigskip

\begin{enumerate}
    \item Jiminez-Vela, Luz
    \item Rabatin, Branislav 
    \item Le, Kan
    \item Stalpes, Kye
    \item Schoedel, Doug
\end{enumerate}

\noindent R. {\bf POSTDOCTORAL SCHOLARS SUPERVISION}. Please list all
postdoctoral scholars whose work you directed during 2020.  {\bf Include name,
date joined group, and residency(i.e. PHYSICS, MAGLAB, CERN, POINT NEMO). }
\bigskip

\noindent S. OUTREACH ACTIVITIES. Please list all outreach
activities that you have participated in during 2020.
\bigskip

My primary outreach activities are \emph{Ask a Scientist} which takes place at
\emph{First Friday}, at the Railyard, and \emph{Saturday Morning Physics}, which
took place on campus at FSU.  Both of these activities were suspended for 2020
due to the pandemic.  

I participated in the online reboot of \emph{Ask a Scientist}, which takes place
on \emph{Twitch}.  Here we played video games and answered questions from the
stream audience.

\noindent T. OTHER ITEMS. Please detail any other items that you
feel will help give an adequate picture of your performance during
2020.
\bigskip

I wrote the liner notes to a new novelty record about outer space, \emph{Race to
Space} by the band ``Satanic Pupeteer Orchestra.''  It's vaguely science
related.

\noindent U.  SPCI EVALUATIONS. Please attach a copy of all student
evaluation SPCI form summary pages for your teaching during 2020.


\end{document}




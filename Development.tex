
\documentclass[10pt]{article}
\topmargin -.5in
\textwidth 6.5in
\textheight 9in
\oddsidemargin 0in

\pagestyle{empty}

\begin{document}

== The major things I've developed: ==

AMR-MHD:  This uses Balsara's 2001 AMR technique (and of course Enzo).
 Initially we used Ryu and Jones (1995) and their CT (1998), though
 after some initial turbulence tests we tried also Balsara Spicer CT
 (1999).  This failed to get a high enough resolution, though the AMR
 seem to be functional.

 Unsplit Second Order MHD: Currently unpublished.  This used MUSCL
 Hancock reconstruction, second order Runge Kutta time integration, and
 a variety of slope limiters and reconstruction fields (primitive,
 conserved, characteristic) and the Gardiner \& Stone (2005) CT
 techniques.  Currently has unresolved stability issues, though many
 stability improvements were made over the course of it's development,
 including unsplit Lapidus viscosity and active reimann solver switches
 based on failure detection.

 Grahical IDL analysis and visualization package, "The Microscope."
 The Microscope is an object oriented IDL tool centered around not only
 visualizing AMR data, but comparison of different datasets and
 snapshots.  Development of the Microscope centered heavily on detailed
 debugging of numerical MHD methods.

 Run Tracker: A system for cataloging and examining history and
 performance of simulations.  The system posts information about a run
 (e.g. name, description, number of nodes/cores) at the submission of
 the job to a database, updates on it's start, and upon completion
 posts statistics about the run itself (e.g. total wall time/SUs,
 start/end cycle, start/end simulation time) This helps the scientist
 keep track of suites of simulations as well as easily record progress.

 Code Flow: Automated flow chart.  A suite of python code analyzed the
 call history for a software package (order of function calls and calls
 stacks) and reported through an AJAX (Dynamic HTML) front end that
 allowed the developer to easily view each function, it's place in all
 used call stacks, and all the function it in turn called.

 Additionally I've written many documents outlining the underpinnings
 of Enzo as well as user guides for beginning development.

\end{document}

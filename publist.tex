\newcommand{\citeform}[1]{{\bf (#1 citations)}}
%\newcommand{\citeform}[1]{}

\medskip
\noindent
``The Power Spectra of Polarized, Dusty Filaments'', Huffenberger, K.; Rotti,
A., Collins, D. C.
\citeform{2}%2019arXiv190610052H

\medskip
\noindent
``The Impact of Enhanced Halo Resolution on the Simulated Circumgalactic Medium''
Hummels, Cameron B.; Smith, Britton D.; Hopkins, Philip F.; O'Shea, Brian W.; Silvia, Devin W.; 
Werk, Jessica K.; Lehner, Nicolas; Wise, John H.; Collins, David C.; Butsky, Iryna S.   
Arxiv 18111241, 2018
\citeform{10}%2018arXiv181112410H

\medskip
\noindent
``Near-infrared Spectral Evolution of the Type Ia Supernova 2014J in the Nebular
Phase: Implications for the Progenitor System''
Diamond, T. R.; Hoeflich, P.; Hsiao, E. Y.; Sand, D. J.; Sonneborn, G.;
Phillips, M. M.; Hristov, B.; Collins, D. C.; Ashall, C.; Marion, G. H.;
Stritzinger, M.; Morrell, N.; Gerardy, C. L.; Penney, R. B.
ApJ, 2018, 861, 119
\citeform{7}%2018ApJ...861..119D

\medskip
\noindent
``Magnetohydrodynamical Effects on Nuclear Deflagration Fronts in Type Ia Supernovae''
Hristov, Boyan; Collins, David C.; Hoeflich, Peter; Weatherford, Charles A.; Diamond, Tiara R., 
ApJ, 2018, 858, 13
%This paper is the first look into the effects of magnetic fields on nuclear burning in Type Ia Supernovae. Dr. Hristov was a PhD student at FAMU that I co-advised with Charles Weatherford, also at FAMU; Dr. Diamond is a postdoc at NASA Goddard, who got her PhD at FSU; Prof. Hoeflich is in physics at FSU; Prof. Weatherford is in physics at FAMU
\citeform{6}%2018ApJ...858...13H


\medskip
\noindent
``The Anatomy of the Column Density Probability Distrbution Function (N-PDF)''
Chen, H., Burkhart, B., Goodman, A. A., Collins, D. C. 
ApJ, 2018, 859, 162
%We compared my simulations to  observations of star-forming clouds in order to understand how the complex structure of these clouds regulates star formation.  H. Chen was a graduate student at Harvard; Dr. Burkhart was a postdoc at Harvard; Prof. Goodman is faculty at Harvard
\citeform{8}%2018ApJ...859..162C

\medskip
\noindent
``Signatures of progenitors of Type Ia supernovae''
Hoeflich, P.; Chakraborty, S.; Comaskey, W.; Fisher, A.; Hristcov, B.; Collins, D.; Diamond, T. R.; Dragulin, P.; Hsiao, E. Y.; Sadler, B.
Conference, Memorie della Societa Astronomica Italiana, 2017, v.88, p.302
%We explored the effects of strong magnetic fields on the spectra of Type Ia supernovae.  Profs. Hoeflich and Hsiao are in physics at FSU; everyone else on the paper was a grad student in physics at FSU.
\citeform{1}%2017MmSAI..88..302H

\medskip
\noindent
``GMC Collisions as Triggers of Star Formation. III. Density and Magnetically Regulated Star Formation''
Wu, B., Tan, J. C., Christie, D., Nakamura, F., Van Loo, S., Collins, D. C. 
ApJ, 2017, 841, 888
%We simulated clouds of star forming gas colliding to explore the behavior of magnetic fields on their gravitational collapse.  Prof. Tan was in physics at University of Florida; Dr. Wu was a graduate student at UF; Dr. Christie was a postdoc at UF; Dr. Van Loo was a postdoc at Leeds; Prof. Nakamura is at National Astronomical Observatory, Mitaka, Tokyo
\citeform{20}%2017ApJ...841...88W

\medskip
\noindent
``The Razor’s Edge of Collapse: The Transition Point from Lognormal to Power-Law Distributions in Molecular Clouds,'' 
Burkhart, B., Stalpes, K., Collins, D. C.,
ApJ, 2017, 834,1
%The statistical properties of structures in star forming clouds can give hints about why stars form so much slower than expected.  Here we explored one of the statistical properties as it relates to gravitational collapse.  Dr. Burkhart was a postdoc at Harvard; K. Stalpes is a graduate student in physics at FSU.
\citeform{18}%2017ApJ...834L...1B

\medskip
\noindent
``GMC Collisions as Triggers of Star Formation. II. 3D Turbulent, Magnetized Simulations'' 
Wu, B., Tan, J. C., Nakamura, F., Van Loo, S., Christie, D., Collins, D. C., 
ApJ, 2017, 835, 137
%We simulated clouds of star forming gas colliding to explore the observational properties of such collapse.  Prof. Tan was in physics at University of Florida; Dr. Wu was a graduate student at UF; Dr. Christie was a postdoc at UF; Dr. Van Loo was a postdoc at Leeds; Prof. Nakamura is at National Astronomical Observatory, Mitaka, Tokyo
\citeform{19}%2017ApJ...835..137W

\medskip
\noindent
``Matching dust emission structures and magnetic field in high-latitude cloud L1642: comparing Herschel and Planck maps,''
Malinen, J.; Montier, L.; Montillaud, J.; Juvela, M.; Ristorcelli, I.; Clark, S.  E.; Bern\'e, O.; Bernard, J.-Ph.; Pelkonen, V.-M.; Collins, D. C. 
MNRAS, 2016, 460, 1934
%We examined the properties of a particularly interesting star-forming cloud using many techniques.  Dr. Malinen was a postdoc working with me at FSU; the others are external collaborators.
\citeform{22}%2016MNRAS.460.1934M


\medskip
\noindent
``Length Scales and Turbulent Properties of Magnetic Fields in Simulated Galaxy Clusters,'' 
Egan, O’Shea, Hallman, Burns, Xu, Collins, Li, Norman  
submitted to ApJ, 2016, arxiv 1601.05083
%The gas within galaxy clusters is magnetized and in a state that is somewhere between gas and plasma; we examined the statistics of turbulence in this gas.  All co-authors were graduate students, post-docs, or faculty at external universities.
\citeform{6}%2016arXiv160105083E

\medskip
\noindent
``Self-Generated Turbulence in Magnetic Reconnection'', 
Oishi, J. S., Mac Low, M., Collins, D. C., Tamura, M.  
ApJ Letters, 2015, 120
%Magnetic fields in fluids can sometimes snap like rubber bands, and immediately re-connect with other nearby fields.  This process, often observed in the sun, is not well understood, and we used computer simulations to explore the production of turbulence in the process.  Profs. Oishi, Mac Low, and Tamura were at Farmingdale State College, the American Natural History Museum, and Barnard, respectively.
\citeform{23}%2015ApJ...806L..12O

\medskip
\noindent
``Observational Diagnostics of Self­Gravitating MHD Turbulence in Giant Molecular Clouds'', 
Burkhart, B., Collins, D.  C., Lazarian, A.
ApJ,  2015, 808, 48
%We explored an array of statistical measurements to relate properties of star-forming clouds that are easily observed to properties that are difficult or impossible to observe.  Dr. Burkhart was at Harvard, and Prof. Lazarian is at University of Wisconsin Madison.
\citeform{41}%2015ApJ...808...48B

\medskip
\noindent
``Enzo: An Adaptive Mesh Refinement Code for Astrophysics'', 
Bryan, G. L. , Norman, M. L. , O'Shea, B. W. , Abel, T. , Wise, J. H. , Turk, M. J. , Reynolds, D. R. , Collins, D. C. , Wang, P. , Skillman, S. W. , Smith, B. , Harkness, R. P. , Bordner, J. , Kim, J.-h. , Kuhlen, M. , Xu, H. , Goldbaum, N. , Hummels, C. , Kritsuk, A. G. , Tasker, E., Skory, S. , Simpson, C. M. , Hahn, O. , Oishi, J. S. , So, G. C. , Zhao, F. , Cen, R. , Li, Y. D,
Astrophysical Journal Suppliment, 2014, 211
%Enzo is a large and very popular software package for simulating astrophysical phenomenon.  This paper presents the details of the algorithms and behavior of the code.  All co-authors are at external institutions.
\citeform{354}%2014ApJS..211...19B

\medskip
\noindent
``Local Support Against Gravity in Magnetoturbulent Fluids'',
Schmidt, W., Collins, D. C., Kritsuk, A. G.,
MNRAS, 2013, 43, 
\citeform{19}%2013MNRAS.431.3196S


\medskip
\noindent
``Cosmological MHD Simulations of Galaxy Cluster Radio Relics: Insights and Warnings for Observations'', 
Skillman, S. W., Xu, H., Hallman, E. J., O’Shea, B. W., Burns, J., Li, H., Norman, M. L., Collins, D. C.,
ApJ, 2013, 765, 21
\citeform{77}%2013ApJ...765...21S

\medskip
\noindent
``Comparisons of Cosmological Magnetohydrodynamic Galaxy Cluster Simulations to Radio Observations'', 
Xu, H., Govoni, F., Murgia, M., Li, H., Collins, D. C., Norman, M. L., Cen, R., Feretti, L., Giovannini, G.,
ApJ, 2012, 759, 40
\citeform{27}%2012arXiv1209.2737X

\medskip
\noindent
``The Two States of Star Forming Clouds'', 
Collins, D.~C., Kritsuk, A., Padoan, P., Li, H., Xu, H., Ustyugov, S., Norman, M.~L.,
ApJ, 2012, 750, 13
\citeform{75}%2012ApJ...750...13C


\medskip
\noindent
``Comparing Numerical Methods for Isothermal Magnetized Supersonic Turbulence'', 
Kritsuk, A., Nordlund, A, Collins, D.~C., et al.,
ApJ, 2011, 737, 13
\citeform{81}%2011ApJ...737...13K

\medskip
\noindent
``Accuracy of Core Mass Estimates in Simulated Observations of Dust Emission'', 
Malinen, J., Juvela, M., Collins, D. C., Lunttila, T., Padoan, P.,
A\&A, 2011, 530, A101
\citeform{55}%2011A&A...530A.101M

\medskip
\noindent
``Mass and Magnetic Distributions in Self Gravitating Super Alfv\' enic Turbulence with AMR'', 
Collins, D.~C., Padoan, P., Norman, M.~L., Xu, H.,
ApJ, 2011, 731, 59
\citeform{40}%2011ApJ...731...59C

\medskip
\noindent
``Evolution and Distribution of Magnetic Fields from Active Galactic Nuclei in Galaxy Clusters. II. The Effects of Cluster Size and Dynamical State",
Xu, H., Li, H., Collins, D.~C., Li, S., Norman, M.~L.,
ApJ, 2011, 739, 77
\citeform{36}%2011ApJ...739...77X

\medskip
\noindent
``Evolution and Distribution of Magnetic Fields from AGNs in Galaxy Clusters.  I.  The Effect of Injection Energy and Redshift'', 
Xu, H., Li, H., Collins, D. C., Li, S., Norman, M.~L.,
ApJ, 2010, 752, 2152
\citeform{46}%2010ApJ...725.2152X

\medskip
\noindent
``The Effect of Projection on Derived Mass-Size and Linewidth-Size Relationships'', 
Shetty, R., Collins, D.~C.,  Kauffmann, J.,  Goodman, A.~A.,  Rosolowsky, E.~W.,  Norman, M.~L., 
ApJ, 2010, 712, 1049
\citeform{48}%2010ApJ...712.1049S

\medskip
\noindent
``Cosmological AMR MHD with Enzo'', 
Collins, D.~C., Xu, H., Norman, M.~L., Li, H., Li., S.,
ApJ Suppliment, 2010, 186, 308
\citeform{67}%2010ApJS..186..308C

\medskip
\noindent
``Turbulence and Dynamo in Galaxy Cluster Medium: Implications on the Origin of Cluster Magnetic Fields'',
Xu, H., Li, H., Collins, D.~C., Li, S., Norman, M.~L.,
ApJ, 2009, 698, L14
\citeform{67}%2009ApJ...698L..14X

\medskip
\noindent
``The Biermann Battery in Cosmological MHD Simulations of Population III Star Formation'',
Xu, H., O'Shea, B. W., Collins, D.~C., Norman, M.L.,
ApJ Letters, 2008, 688, L57  
\citeform{70}%2008ApJ...688L..57X

\medskip
\noindent
``Formation of X-Ray Cavities by the Magnetically Dominated Jet-Lobe System in a Galaxy Cluster'',
Xu, H., Li., H., Collins, D. C., Li, S., Norman, M. L.,
ApJ Letters, 2008, 681, L61 
\citeform{19}%2008ApJ...681L..61X

\documentclass[10pt]{article}
\topmargin -.5in
\textwidth 6.5in
\textheight 9in
\oddsidemargin 0in

\pagestyle{empty}
\begin{document}
\newcommand{\LT}[1]{\noindent\textbf{#1}}
\begin{LARGE}
\noindent Current and Pending Support David C. Collins
\end{LARGE}


%Project Title:
%Project PI:
%Program Name:
%Point of Contact:
%Period of Performance:
%FTE: 

\vspace{0.1in}

\noindent \large{\textbf{\emph{Current Funding}}}

\vspace{0.1in}

\LT{Project Title:} Magnetic Fields in the Formation of Molecular Clouds,
Filaments, and Cores

\LT{Project PI:} D. C. Collins

\LT{Program Name and award number:} NSF AAG  AST-1616026 

\LT{Period of Performance:} 09/01/2016 - 08/31/2019

\LT{Amout} \$298,492 

\LT{FTE:} 1.0 Month/year
%\LT{FTE:} 1.5 Month/year

\LT{Summary of Work:} This project is studying the gravitational collapse of
molecular clouds using simulations of magnetohydrodynamical turbulence.  There
is a synergistic overlap with the proposed work in the shared use of high
performance computing software and resources.

\vspace{0.1in}


\LT{Project Title:} Modeling CMB polarization foregrounds and their isotropy
violation

\LT{Project PI:} K. Huffenberger 

\LT{Program Name and award number:} NASA ATP  NNX17AF87G

\LT{Period of Performance:} 01/08/2017 - 01/07/2020

\LT{Amount} \$428,043.00

\LT{FTE:} 1.0 Months/Year

\LT{Summary of Work:} In this project, we are modeling the CMB foregrounds using
a number of techniques, including an analytic filament model and simulations.
The proposed work continues and extends the researching ongoing in this project.

\vspace{0.1in}

\LT{Project Title:} Signatures of Type Ia Supernovae Explosions and their
Cosmological Implications

\LT{Project PI:} P. Hoeflich 

\LT{Program Name and award number:} NSF AAG AST-1715133

\LT{Period of Performance:} 10/01/2017 - 09/31/2020

\LT{Amount} \$460,498.00

\LT{FTE:} 0.0 Months/Year

\LT{Summary of Work:} In this project, we examine the role of magnetic fields in
Type Ia supernovae.  There
is a synergistic overlap with the proposed work in the shared use of high
performance computing software and resources.


\vspace{0.1in}

\noindent \large{\textbf{\emph{Pending}}}

\LT{Project Title:} SNE Ia: Imprints of the Explosion \& Progenitor,
Model-Independent Relations \& Cosmology.

\LT{Project PI:} P. Hoeflich 

\LT{Program Name:} NSF AAG 

\LT{Period of Performance:} 8/1/2020-7/31/2023  

\LT{Amount} \$498,209

\LT{FTE:} 0.5 Months/Year

\LT{Summary of Work:} This proposal will examine a new model for Type Ia
Supernovae light curves.
In addition we will examine the impact of magnetic fields on Type Ia light
curves.

\vspace{0.1in}

\LT{Project Title:} CMB Polarization Foreground Effects on B-modes and Lensing

\LT{Project PI:} K. Huffenberger    

\LT{Program Name:} NSF AAG 

\LT{Period of Performance:} 8/1/2020-7/31/2023  

\LT{Amount} \$533,715

\LT{FTE:} 1. Months/Year

\LT{Summary of Work:} 
This proposal is a general study of foreground effects on CMB-based
gravitational lensing products. It proposes to use analytic filament models and MHD
simulations, similar to the current proposal. It differs from the current proposal in that there is
no postdoc funding to examine delensing and its specific effect on CMB-S4's science goals.

\vspace{0.1in}

\LT{Project Title:} Simulations of Galactic Magnetism

\LT{Project PI:} D. Collins

\LT{Program Name:} NSF AAG 

\LT{Period of Performance:} 8/1/2020-7/31/2023  

\LT{Amount} \$282,024

\LT{FTE:} 1. Months/Year

\LT{Summary of Work:} This proposal will produce simulations of the Galactic
magnetic field.  These galaxy simulations will offer self-consistent magnetic
pictures, while the current proposal will offer high resolution and smaller
scale structures.  


\vspace{0.1in}
\end{document}
